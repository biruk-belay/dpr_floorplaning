\section{Floorplanning problem formulation}

\textbf{\\ put a picture of PR design flow in Xilinx} \\

Partial-reconfiguration involves dynamically switching modules in a reconfigurable region whilst other reconfigurable and static regions continue to be operational. \textbf{Fig} depicts the PR design flow in Xilinx FPGAs as implemented in Xilinx Vivado. The major steps in the automated PR flow are \\

\begin{itemize}
\item \textit{Synthesis}: At this level the behavioral description of modules written in hardware description langauges (Verilog or VHDL) is converted into a gate-level netlist. In PR design flow, floorplanning is mandatory (it is optional in the standard flow) and it must be done before the implementation step.  
\item \textit{Implementation}: In this step the gate-level netlist output of the previous stage is functionally mapped to specific device resources on the FPGA and then placed and routed.
\item \textit{Bitstream generation}: In PR desgin flow, the final output of an FPGA design tool is a set of bit files which contain the configuration information for both the logic and routing resources of the static and reconfigurable regions. 
\end{itemize} 

\hfill \break

In floorplanning for PR, each R$_i$, which are also named \textit{Pblocks} in the Xilinx design flow, are subject to the following restrictions and requirements. 

\begin{enumerate}
\item Reconfigurable regions must contain only valid reconfigurable resources (for example in 7 series FPGAs only CLBs, BRAMs and DSPs must be in R$_i$)
\item The minimum number of resources incorporated by a reconfigurable region \textbf{R$_i$} (Pblock) must at least be equal to the resource requirement of the largest module hosted in the region

\item Two reconfigurable regions must not overlap. Overlapping is equivalent to sharing at least a single tile

\item The left and right edges of R$_i$ must not split interconnect columns 

\item Reconfigurable regions(Pblocks) can span non reconfigurable components such as configuration blocks, central clock column etc... without considering them as part of the region(Pblocks)

\item The height of R$_i$ must be aligned to clock regions. This is an optional requirement but enforcing it results in starting the reconfigurable module from a known initial state re-configuration

\end{enumerate}

As described in the previous section the proposed floorplanner takes as an input the resource description of the FPGA fabric and the resource requirement of each reconfigurable region and produces a feasible placement for each region.


\begin{comment}
The central clock column divides the FPGA into left and right regions as shown in the figure. But all the horizontally adjacent clock regions are combined into a single clock region to make the x axis W units wide. The y axis is H units high. Combining horizontally adjacent clock regions into a single clock region does

Reconfigurable regions need not necessarily be rectangular but assigning rectangular shapes to regions greatly reduces routing and placement challenges

The bit files are deployed in the FPGA configuration memory. 
\end{comment}