\section{MILP formulation}
In this section the MILP model of the PR floorplanning problem is presented. The section first summarizes the considered assumptions and abstractions related both to the FPGA and its resources as well as the floorplanning problem. The description of the model consists of defining variables, constraints and an objective function to be optimized. 

% also add how the central clock column is modeled as a forbidden region}. \\

%The total number of a resource R within a slot S$_i$ = (x$_i$, y$_i$, w$_i$, h$_i$) is then equal to
%\begin{equation}
%R = (x_i + w_i) \cdot (y_i + h_i)
%\end{equation}  



\subsection{definition of optimization variables}
\begin{comment}
To encode the MILP formulation the following binary and real variables are defined. \\ \\
Variables related to the size, location and number of resources in slot S$_i$. \\
For each slot S$_i$
\begin{itemize}
\item N: set of reconfigurable regions

\item S$_i$: slot i $\mid$ S$_i$ $\in$ N 

\item W $\in$ $\mathbb{Z}$ represents the total width of the fpga fabric

\item H $\in$ $\mathbb{Z}$ specifies the height of the fpga fabric

\item C: the set of clock regions on the fpga

\item r $\in$ $\mathbb{Z}$ denotes the number of rows inside a single clock region

\item F: set of forbidden regions

\item F$_k$: forbidden region k $\in$ F

\item x$_i$, y$_i$ w$_i$, h$_i$ $\in$ $\mathbb{Z}$ represent the bottom left coordinates, the width and the height of S$_i$  respectively
\end{itemize}

\begin{itemize}
\item clb$_i$, bram$_i$ and dsp$_i$  $\in$ $\mathbb{Z}$ represent the number of clb, bram and dsp between x$_i$ and x$_i$ + w$_i$ in a single row respectively.
\end{itemize}

\begin{itemize}
\item clb$\_$req$_i$, bram$\_$req$_i$ and dsp$\_$req$_i$ $\in$ $\mathbb{Z}$ represent the required number of clb, bram and dsp in S$_i$.
\end{itemize}

\begin{itemize}
\item $\beta_{ijk}$ $\in$ [0,1] represents row k in clock region j for slot S$_i$.
\end{itemize}

\end{comment}

\hfill \break
Variables denoting the relationship between two slots 

For two slots S$_i$ and S$_k$ 
\begin{itemize}
\item $\gamma_{ik}$ $\in$ [0,1] is a binary variable used to identify whether S$_i$ is found on the left or on the right of S$_k$\\
$\gamma_{ik}$ = 1 if x$_i$ $\leq$ x$_k$ [i.e. S$_i$ is on the left of S$_k$]

\begin{comment}
\item $\theta_{ik}$ $\in$ [0,1] is a binary variable used to identify whether S$_i$ is found on the top or bottom of of S$_k$\\
$\theta_{ik}$ = 1 if y$_i$ $\leq$ y$_k$ [i.e. S$_i$ is found below S$_k$]

\item $\Gamma_{ik}$ $\in$ [0,1] is used to denote if bottom right x coordinate of S$_i$ is found to the right of the bottom left coordinate of S$_k$ \\
$\Gamma$ = 1 if x$_i$ + w$_i$ $\geq$ x$_k$

\item $\eta_{ik}$ $\in$ [0,1] is used to denote if bottom right x coordinate of S$_k$ is found to the right of the bottom left coordinate of S$_i$ \\
$\eta$ = 1 if x$_k$ + w$_k$ $\geq$ x$_i$

\item $\Omega_{ik}$ $\in$ [0,1] is used to denote if the top y coordinate of S$_i$ is found above the lower y coordinate of S$_k$ \\
$\Omega$ = 1 if y$_i$ + h$_i$ $\geq$ y$_k$

\item $\Psi_{ik}$ $\in$ [0,1] is used to denote if top y coordinate of S$_k$ is found above the lower y coordinate of S$_i$ \\
$\Psi$ = 1 if y$_k$ + h$_k$ $\geq$ y$_i$


\item $\Delta_{ik}$ $\in$ [0,1] is a binary variable which indicates interfernce between slots S$_i$ and S$_k$.\\
$\Delta_{ik}$ = 0 if there is no interference between the slots [i.e. not a single tile is shared between slots]

\end{comment}
%\item D$_{w}$, B$_w$ and C$_w$ represent wasted DSPs, BRAMs and CLBs in S$_i$ respectively
%\item $\alpha_i$ is a real variable that is used to express the bound on the amount of wasted resources in a slot S$_i$
%\item $\rho$ and $\nu$ are vectors which contain the x and y coordinates of all the forbidden columns and rows on the FPGA fabric respectively 
%\item $\eta_{ik}$ is a real variable that expresses the bound on the wirelength between S$_i$ and S$_k$
\end{itemize}

\hfill \break

\begin{comment}
Variables denoting the relationship between S$_i$ and forbidden region F$_k$.  
\begin{itemize}
\item $\mu_{ik}$ $\in$ [0,1] is a binary variable used to identify whether S$_i$ is found on the left or on the right of F$_k$\\
$\mu_{ik}$ = 1 if x$_i$ $\leq$ fx$_k$ [i.e. S$_i$ is on the left of F$_k$]

\item $\nu_{ik}$ $\in$ [0,1] is a binary variable used to identify whether S$_i$ is found on the top or bottom of of F$_k$\\
$\nu_{ik}$ = 1 if y$_i$ $\leq$ fy$_k$ [i.e. S$_i$ is found below F$_k$]

\item fbdn$_1$ $\in$ [0,1] is used to denote if bottom right x coordinate of S$_i$ is found to the right of the bottom left coordinate of F$_k$ \\
fbdn$_1$ = 1 if x$_i$ + w$_i$ $\geq$ fx$_k$

\item fbdn$_2$ $\in$ [0,1] is used to denote if bottom right x coordinate of F$_k$ is found to the right of the bottom left coordinate of s$_i$ \\
fbdn$_2$ = 1 if fx$_k$ + fw$_k$ $\geq$ x$_i$

\item fbdn$_3$ $\in$ [0,1] is used to denote if the top y coordinate of S$_i$ is found above the lower y coordinate of F$_k$ \\
fbdn$_3$ = 1 if y$_i$ + h$_i$ $\geq$ fy$_k$

\item fbdn$_4$ $\in$ [0,1] is used to denote if top y coordinate of F$_k$ is found above the lower y coordinate of S$_i$ \\
fbdn$\_4$ = 1 if fy$_k$ + fh$_k$ $\geq$ y$_i$
\end{itemize}

\hfill \break
\end{comment}
\subsection{Constraint definition}

\begin{comment}
Slots for partial reconfiguration should fulfill the following constraints
\begin{itemize}
\item there must be enough resources within the slots
\item A frame can not be shared between two reconfigurable partitions (no interference)
\item static resources on the FPGA must not be included in the slots 
\item Left and right edges of slots must be placed in proper positions
\item the amount of wasted resources should be minimized (Wasting DSPs is more expensive than BRAMs which in turn is more expensive than CLBs)
\item Other optimizations such as lower wire length between slots or lower length to I/O etc... can be added as constraints
\end{itemize}

\end{comment}
\hfill \break

\subsubsection{\textbf {Semantics constraints}}
The following constraints ensure the soundness of some of the variables.\\
\begin{constraint} $\forall$ S$_1$ $\in$ N, $\forall$ i = 1...,N\textsuperscript{max} , $\forall$ x$_i$ = 0..., W, $\forall$ y$_i$ = 0..., H $\forall$ w$_i$, $\forall$ h$_i$    
\begin{equation}
\begin{split}
x_i + w_i \leq W \\
y_i + h_i \leq H \\
\end{split}
\end{equation} 
\end{constraint}
\begin{defn} the right most x coordinate and the top y coordinates of S$_i$ must not exceed the boundaries of the fabric \\
\end{defn}

\begin{constraint} $\forall$ i = 1...,N\textsuperscript{max}, $\forall$ j = 1...,clk$\_$reg\textsuperscript{max}, $\forall$ k = 1...,r $\forall$ h$_i$
\begin{equation}
\begin{split}
 h_i = \sum_{j=1}^{clk\_reg} \sum_{k=1}^{r} \beta_{ijk}
\end{split}
\end{equation}
\end{constraint}

\begin{defn}
The height of S$_i$ must be the sum of binary rows in each clock region which are set to 1
\end{defn}

\begin{constraint} $\forall$ i = 1...,N\textsuperscript{max}, $\forall$ j = 1...,clk$\_$reg\textsuperscript{max}, $\forall$ k = 1...,r $\forall$ h$_i$
\begin{equation}
\begin{split}
y_i \leq \sum_{j=1}^{clk\_reg} \sum_{k=1}^{r} H - \beta_{ijk} \cdot (H - (k + (r - 1) \cdot j))
\end{split}
\end{equation}
\end{constraint}
\begin{defn}
y$_i$ must be constrained not to be greater than the lowest chosen row 
\end{defn}


\begin{constraint} $\forall$ i = 1...,N\textsuperscript{max}, $\forall$ j = 1...,clk$\_$reg\textsuperscript{max}, $\forall$ k = 1...,r
\begin{equation}
\begin{split}
\beta_{ij(k+1)} \geq \beta_{ijk} + \beta_{ij(k+2)} - 1 
\end{split}
\end{equation}
\end{constraint}

\begin{defn}
rows in the same clock region in S$_i$ must be contigious i.e., if $\beta_{ij0}$ = 1 \& $\beta_{ij2}$ = 1 then$\beta_{ij1}$ must also be equal to 1.
\end{defn}


\subsubsection{\textbf{Resource constraints}} 
These set of constraints ensure that each slot satisfies the resource requirements of the application. The total number of clbs in a slot S$_i$ is expressed as 

\begin{equation}
CLB(x_i,y_i,w_i,h_i) = \sum_{j=1}^{clk\_reg} \sum_{k=1}^{r} \beta_{ijk} \cdot clb_i 
\label{tot_clb}
\end{equation}

In the same way the amount of bram and dsp can also be expressed as 

\begin{equation}
BRAM(x_i,y_i,w_i,h_i) = \sum_{j=1}^{clk\_reg} \sum_{k=1}^{r} \beta_{ijk} \cdot bram_i 
\label{tot_bram}
\end{equation}


\begin{equation}
DSP(x_i,y_i,w_i,h_i) = \sum_{j=1}^{clk\_reg} \sum_{k=1}^{r} \beta_{ijk} \cdot dsp_i 
\label{tot_dsp}
\end{equation}

The resource constraint can simply be stated as the required number of clbs, brams and dsps must be greater than or equal to CLB(x$_i$,y$_i$,w$_i$,h$_i$), BRAM(x$_i$,y$_i$,w$_i$,h$_i$) and DSP(x$_i$,y$_i$,w$_i$,h$_i$) respectively. But the above functions are non linear and can not be used directly to formulate linear constraints.\\

\subsubsection*{\textit {linearization}}
In order to employ \ref{tot_clb}, \ref{tot_bram} and \ref{tot_dsp} as linear constraint, they must first be linearized. To linearize these functions we define three auxilary real variables $\tau1_{ijk}$, $\tau2_{ijk}$ and $\tau3_{ijk}$. \\

$\tau1_{ijk}$ $\in$ $\mathbb{R}$ $\mid$ $\tau1_{ijk}$ = $\beta_{ijk}$ $\cdot$ clb$_i$ \\

$\tau2_{ijk}$ $\in$ $\mathbb{R}$ $\mid$ $\tau2_{ijk}$ = $\beta_{ijk}$ $\cdot$ bram$_i$ \\

$\tau3_{ijk}$ $\in$ $\mathbb{R}$ $\mid$ $\tau3_{ijk}$ = $\beta_{ijk}$ $\cdot$ dsp$_i$ \\

Hence CLB(x$_i$,y$_i$,w$_i$,h$_i$), BRAM(x$_i$,y$_i$,w$_i$,h$_i$) and DSP(x$_i$,y$_i$,w$_i$,h$_i$) can be restated as 

\begin{equation}
CLB(x_i,y_i,w_i,h_i) = \sum_{j=0}^{clk\_reg} \sum_{k=0}^{r} \tau1_{ijk}
\end{equation}


\begin{equation}
BRAM(x_i,y_i,w_i,h_i) = \sum_{j=0}^{clk\_reg} \sum_{k=0}^{r} \tau2_{ijk}
\end{equation}

\begin{equation}
DSP(x_i,y_i,w_i,h_i) = \sum_{j=0}^{clk\_reg} \sum_{k=0}^{r} \tau3_{ijk}
\end{equation}

Now the non linear expression is replaced by a linear one and to complete the linearlization a few constraints must be set. The following are constraints related to $\tau1_{ijk}$ but similar constraints can be set for $\tau2_{ijk}$ and $\tau3_{ijk}$.

\begin{constraint} $\forall$ i = 1...,N\textsuperscript{max}, $\forall$ j = 1...,clk$\_$reg\textsuperscript{max}, $\forall$ k = 1...,r
\begin{equation}
\begin{split}
\tau1_{ijk} & \geq 0 \\
\tau1_{ijk} & \leq BIG\_M \cdot \beta_{ijk} \\ 
\tau1_{ijk} & \leq clb_i \\
\tau1_{ijk} & \geq clb_i - (1 - \beta_{ijk}) \\
clb\_req_i  & \geq \sum_{j=1}^{clk\_reg} \sum_{k=1}^{r} \tau1_{ijk}
\end{split}
\end{equation}
\end{constraint}


\subsubsection{\textbf{Non-interference constraints}}
\begin{comment}
\subsubsection*{\textit {clock region aligned boundary}}  
A frame (tile) is the smallest reconfigurable physical region and it spans one clock region high and one resource type wide. A reconfigurable frame can not contain logic from more than one reconfigurable partition hence the boundaries of a slot S$_i$ must be forced to fit in to clock region boundaries. 
Such a constraint is enforced by forcing all the rows in a clock region to also be included in the slot if atleast one is included i.e., if one row in a clock region j is part of a slot then the remaining rows within the same clock region j must also be forced to be part of the same slot to satisfy this constraint. \\
To help us set this constraint we define an intermediate variable \\
l$_{j}$  $\in$ $\mathbb{Z}$ $\mid$ $\forall$ j = 1...,clk$\_$reg\textsuperscript{max}, $\forall$ k = 1...,r
\begin{equation}
\begin{split}
l_j = \sum_{k=1}^{r} \beta_{ijk}
\end{split}
\end{equation}

then the following constraint will force the slot boundaries to be aligned with the clock region boundaries by forcing $\beta_{ijk}$ to become part of the slot if at least one row with in the same clock region becomes part of slot S$_i$

\begin{constraint}  $\forall$ i = 1...,N\textsuperscript{max}, $\forall$ j = 1...,clk$\_$reg\textsuperscript{max}, $\forall$ k = 1...,r
\begin{equation}
\begin{split}
\beta_{ijk} \geq \sum_{k=0}^{r}  (l_j - beta_{ijk}) / (r - 1)
\end{split}
\end{equation}
\end{constraint}
\end{comment}

\subsubsection*{\textit {Interference between two slots}}
Two slots S$_i$ and S$_k$ are said to be non interfering under the following conditions
\begin{algorithmic}
\IF{x$_i$ $\leq$ x$_k$ and y$_i$ $\leq$ y$_k$}
	\STATE x$_i$ + w$_i$ $<$ x$_k$ or y$_i$ + h$_i$ $<$ y$_k$
\ELSIF {x$_i$ $\geq$ x$_k$ and y$_i$ $\geq$ y$_k$}
	\STATE x$_i$ + w$_k$ $<$ x$_i$ or y$_k$ + h$_k$ $<$ y$_i$
\ELSIF {x$_1$ $<$ x$_k$ and y$_i$ $>$ y$_k$}
	\STATE x$_i$ + w$_i$ $<$ x$_k$ or y$_k$ + h$_k$ $<$ y$_i$
\ELSE
	\STATE x$_k$ + w$_k$ $<$ x$_k$ or y$_i$ + h$_i$ $<$ y$_k$
\ENDIF
\end{algorithmic}

This above condition can be encoded into a set of MILP constraints as follows \\
\begin{constraint} S$_i$ $\in$ N and S$_k$ $\in$ N
\begin{equation} 
\begin{split}
\delta_{ik} & \geq \gamma_{ik} + \theta_{ik} + \Gamma_{ik} + \Omega_{ik} - 3 \\
\delta_{ik} & \geq (1 - \gamma_{ik}) + \theta_{ik} + \eta_{ik} + \Omega_{ik} - 3 \\
\delta_{ik} & \geq \gamma_{ik} +(1 - \theta_{ik}) + \Gamma_{ik} + \Psi_{ik} - 3 \\
\delta_{ik} & \geq (1 - \gamma_{ik}) + (1 - \theta_{ik}) + \eta_{ik} + \Psi_{ik} - 3 \\
\delta_{ik} & = 0 \\
\end{split}
\end{equation}
\end{constraint}

\subsubsection{\textit {Interference with Forbidden slots}}

As stated before forbidden regions are also modeled as a normal slots hence the constraint for non intereference between a slot S$_i$ and a forbidden region F$_k$ can be set in the same way as done in the previous constraint formulation between two slots

\begin{constraint} S$_i$ $\in$ N and F$_k$ $\in$ F
\begin{equation}
\begin{split}
\delta_{ik} & \geq \mu_{ik} + \nu_{ik} + fbdn_1 + fbdn_3 - 3 \\
\delta_{ik} & \geq (1 - \mu_{ik}) + \nu_{ik} + fbdn_2 + fbdn_3 - 3 \\
\delta_{ik} & \geq \mu_{ik} + (1 - \nu_{ik}) + fbdn_1 + fbdn_4 - 3 \\
\delta_{ik} & \geq (1 - \mu_{ik}) + (1 - \nu_{ik}) + fbdn_2 + fbdn_4 - 3 \\
\delta_{ik} & = 0 \\
\end{split}
\end{equation}
\end{constraint}